\documentclass[]{book}
\usepackage{lmodern}
\usepackage{amssymb,amsmath}
\usepackage{ifxetex,ifluatex}
\usepackage{fixltx2e} % provides \textsubscript
\ifnum 0\ifxetex 1\fi\ifluatex 1\fi=0 % if pdftex
  \usepackage[T1]{fontenc}
  \usepackage[utf8]{inputenc}
\else % if luatex or xelatex
  \ifxetex
    \usepackage{mathspec}
  \else
    \usepackage{fontspec}
  \fi
  \defaultfontfeatures{Ligatures=TeX,Scale=MatchLowercase}
\fi
% use upquote if available, for straight quotes in verbatim environments
\IfFileExists{upquote.sty}{\usepackage{upquote}}{}
% use microtype if available
\IfFileExists{microtype.sty}{%
\usepackage{microtype}
\UseMicrotypeSet[protrusion]{basicmath} % disable protrusion for tt fonts
}{}
\usepackage{hyperref}
\hypersetup{unicode=true,
            pdftitle={Index to LinkedIn Learning R Courses},
            pdfauthor={Mark Niemann-Ross},
            pdfborder={0 0 0},
            breaklinks=true}
\urlstyle{same}  % don't use monospace font for urls
\usepackage{natbib}
\bibliographystyle{plainnat}
\usepackage{longtable,booktabs}
\usepackage{graphicx,grffile}
\makeatletter
\def\maxwidth{\ifdim\Gin@nat@width>\linewidth\linewidth\else\Gin@nat@width\fi}
\def\maxheight{\ifdim\Gin@nat@height>\textheight\textheight\else\Gin@nat@height\fi}
\makeatother
% Scale images if necessary, so that they will not overflow the page
% margins by default, and it is still possible to overwrite the defaults
% using explicit options in \includegraphics[width, height, ...]{}
\setkeys{Gin}{width=\maxwidth,height=\maxheight,keepaspectratio}
\IfFileExists{parskip.sty}{%
\usepackage{parskip}
}{% else
\setlength{\parindent}{0pt}
\setlength{\parskip}{6pt plus 2pt minus 1pt}
}
\setlength{\emergencystretch}{3em}  % prevent overfull lines
\providecommand{\tightlist}{%
  \setlength{\itemsep}{0pt}\setlength{\parskip}{0pt}}
\setcounter{secnumdepth}{5}
% Redefines (sub)paragraphs to behave more like sections
\ifx\paragraph\undefined\else
\let\oldparagraph\paragraph
\renewcommand{\paragraph}[1]{\oldparagraph{#1}\mbox{}}
\fi
\ifx\subparagraph\undefined\else
\let\oldsubparagraph\subparagraph
\renewcommand{\subparagraph}[1]{\oldsubparagraph{#1}\mbox{}}
\fi

%%% Use protect on footnotes to avoid problems with footnotes in titles
\let\rmarkdownfootnote\footnote%
\def\footnote{\protect\rmarkdownfootnote}

%%% Change title format to be more compact
\usepackage{titling}

% Create subtitle command for use in maketitle
\providecommand{\subtitle}[1]{
  \posttitle{
    \begin{center}\large#1\end{center}
    }
}

\setlength{\droptitle}{-2em}

  \title{Index to LinkedIn Learning R Courses}
    \pretitle{\vspace{\droptitle}\centering\huge}
  \posttitle{\par}
    \author{Mark Niemann-Ross}
    \preauthor{\centering\large\emph}
  \postauthor{\par}
      \predate{\centering\large\emph}
  \postdate{\par}
    \date{2020-01-04}

\usepackage{booktabs}
\usepackage{amsthm}
\makeatletter
\def\thm@space@setup{%
  \thm@preskip=8pt plus 2pt minus 4pt
  \thm@postskip=\thm@preskip
}
\makeatother

\begin{document}
\maketitle

{
\setcounter{tocdepth}{1}
\tableofcontents
}
\hypertarget{introduction}{%
\chapter*{Introduction}\label{introduction}}
\addcontentsline{toc}{chapter}{Introduction}

Hello

\hypertarget{list-of-courses}{%
\chapter*{List Of Courses}\label{list-of-courses}}
\addcontentsline{toc}{chapter}{List Of Courses}

\href{https://linkedin-learning.pxf.io/codeclinic_R}{Code Clinic: R}

\href{https://linkedin-learning.pxf.io/r_dates}{R Programming in Data Science: Dates and Times}

\href{https://linkedin-learning.pxf.io/r_highvariety}{R Programming in Data Science: High Variety Data}

\href{https://linkedin-learning.pxf.io/r_highvelocity}{R Programming in Data Science: High Velocity Data}

\href{https://linkedin-learning.pxf.io/r_highvolume}{R Programming in Data Science: High Volume Data}

\href{https://linkedin-learning.pxf.io/r_lunchbreak}{R for Data Science: Lunchbreak Lessons}

\href{https://linkedin-learning.pxf.io/setupstart}{R Programming in Data Science: Setup and Start}

\hypertarget{a}{%
\chapter*{A}\label{a}}
\addcontentsline{toc}{chapter}{A}

\hypertarget{b}{%
\chapter*{B}\label{b}}
\addcontentsline{toc}{chapter}{B}

\href{https://linkedin-learning.pxf.io/rweekly_atomics}{Basic Data Types}

\hypertarget{a-1}{%
\chapter*{A}\label{a-1}}
\addcontentsline{toc}{chapter}{A}

\hypertarget{d}{%
\chapter*{D}\label{d}}
\addcontentsline{toc}{chapter}{D}

\href{https://linkedin-learning.pxf.io/rwkly_dataSets}{data sets}

\hypertarget{a-2}{%
\chapter*{A}\label{a-2}}
\addcontentsline{toc}{chapter}{A}

\hypertarget{a-3}{%
\chapter*{A}\label{a-3}}
\addcontentsline{toc}{chapter}{A}

\hypertarget{a-4}{%
\chapter*{A}\label{a-4}}
\addcontentsline{toc}{chapter}{A}

\hypertarget{a-5}{%
\chapter*{A}\label{a-5}}
\addcontentsline{toc}{chapter}{A}

\hypertarget{a-6}{%
\chapter*{A}\label{a-6}}
\addcontentsline{toc}{chapter}{A}

\hypertarget{a-7}{%
\chapter*{A}\label{a-7}}
\addcontentsline{toc}{chapter}{A}

\hypertarget{a-8}{%
\chapter*{A}\label{a-8}}
\addcontentsline{toc}{chapter}{A}

\hypertarget{a-9}{%
\chapter*{A}\label{a-9}}
\addcontentsline{toc}{chapter}{A}

\hypertarget{a-10}{%
\chapter*{A}\label{a-10}}
\addcontentsline{toc}{chapter}{A}

\hypertarget{a-11}{%
\chapter*{A}\label{a-11}}
\addcontentsline{toc}{chapter}{A}

\hypertarget{a-12}{%
\chapter*{A}\label{a-12}}
\addcontentsline{toc}{chapter}{A}

\hypertarget{a-13}{%
\chapter*{A}\label{a-13}}
\addcontentsline{toc}{chapter}{A}

\hypertarget{a-14}{%
\chapter*{A}\label{a-14}}
\addcontentsline{toc}{chapter}{A}

\hypertarget{a-15}{%
\chapter*{A}\label{a-15}}
\addcontentsline{toc}{chapter}{A}

\hypertarget{s}{%
\chapter*{S}\label{s}}
\addcontentsline{toc}{chapter}{S}

\href{https://linkedin-learning.pxf.io/rweekly_subset}{Subsetting}

\hypertarget{a-16}{%
\chapter*{A}\label{a-16}}
\addcontentsline{toc}{chapter}{A}

\hypertarget{a-17}{%
\chapter*{A}\label{a-17}}
\addcontentsline{toc}{chapter}{A}

\hypertarget{v}{%
\chapter*{V}\label{v}}
\addcontentsline{toc}{chapter}{V}

\href{https://linkedin-learning.pxf.io/rweekly_vectormath}{Vector Math}

\hypertarget{a-18}{%
\chapter*{A}\label{a-18}}
\addcontentsline{toc}{chapter}{A}

\hypertarget{a-19}{%
\chapter*{A}\label{a-19}}
\addcontentsline{toc}{chapter}{A}

\hypertarget{a-20}{%
\chapter*{A}\label{a-20}}
\addcontentsline{toc}{chapter}{A}

\hypertarget{a-21}{%
\chapter*{A}\label{a-21}}
\addcontentsline{toc}{chapter}{A}


\end{document}
